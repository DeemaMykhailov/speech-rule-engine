
\SEC{Radicals}
\label{sec:radicals}

\R
\E \[\sqrt{2}\]
\begin{longtable}[c]{@{}lll@{}}
\toprule\addlinespace
verbose & StartRoot 2 EndRoot &

\\\addlinespace
brief & StartRoot 2 EndRoot &

\\\addlinespace
superbrief & Root 2 EndRoot &

\\\addlinespace
\bottomrule
\end{longtable}


\E \[\sqrt{m+n}\]
\begin{longtable}[c]{@{}lll@{}}
\toprule\addlinespace
verbose & StartRoot m plus n EndRoot &

\\\addlinespace
brief & StartRoot m plus n EndRoot &

\\\addlinespace
superbrief & Root m plus n EndRoot &

\\\addlinespace
\bottomrule
\end{longtable}


\R
\E \[\sqrt[m+n]{x+y}\]
\begin{longtable}[c]{@{}lll@{}}
\toprule\addlinespace
verbose & RootIndex m plus n StartRoot x plus y EndRoot &

\\\addlinespace
brief & RootIndex m plus n StartRoot x plus y EndRoot &

\\\addlinespace
superbrief & Index m plus n Root x plus y EndRoot &

\\\addlinespace
\bottomrule
\end{longtable}


\E \[\sqrt[n]{x^m}=\left(\sqrt[n]{x}\right)^m=x^{\frac{m}{n}}, x>0\]
\begin{longtable}[c]{@{}lll@{}}
\toprule\addlinespace
verbose & RootIndex n StartRoot x Superscript m Baseline EndRoot equals
left-parenthesis RootIndex n StartRoot x EndRoot right-parenthesis
Superscript m Baseline equals x Superscript StartFraction m Over n
EndFraction Baseline comma x greater-than 0 &

\\\addlinespace
brief & RootIndex n StartRoot x Sup m Base EndRoot equals left-p'ren
RootIndex n StartRoot x EndRoot right-p'ren Sup m Base equals x Sup
StartFrac m Over n EndFrac Base comma x greater-than 0 &

\\\addlinespace
superbrief & Index n Root x Sup m Base EndRoot equals L p'ren Index n
Root x EndRoot R p'ren Sup m Base equals x Sup Frac m Over n EndFrac
Base comma x greater-than 0 &

\\\addlinespace
\bottomrule
\end{longtable}


\E \[\sqrt[3]{x}=x^{\frac{1}{3}}\]
\begin{longtable}[c]{@{}lll@{}}
\toprule\addlinespace
verbose & RootIndex 3 StartRoot x EndRoot equals x Superscript one-third
&

\\\addlinespace
brief & RootIndex 3 StartRoot x EndRoot equals x Sup one-third &

\\\addlinespace
superbrief & Index 3 Root x EndRoot equals x Sup one-third &

\\\addlinespace
\bottomrule
\end{longtable}


\R
\E \[\sqrt{\sqrt{x+1}+\sqrt{y+1}}\]
\begin{longtable}[c]{@{}lll@{}}
\toprule\addlinespace
verbose & NestedStartRoot StartRoot x plus 1 EndRoot plus StartRoot y
plus 1 EndRoot NestedEndRoot &

\\\addlinespace
brief & NestStartRoot StartRoot x plus 1 EndRoot plus StartRoot y plus 1
EndRoot NestEndRoot &

\\\addlinespace
superbrief & NestRoot Root x plus 1 EndRoot plus Root y plus 1 EndRoot
NestEndRoot &

\\\addlinespace
\bottomrule
\end{longtable}


\E \[\sqrt[n]{\sqrt[m]{x}}=\sqrt[m]{\sqrt[n]{x}}\]
\begin{longtable}[c]{@{}lll@{}}
\toprule\addlinespace
verbose & NestedRootIndex n NestedStartRoot RootIndex m StartRoot x
EndRoot NestedEndRoot equals NestedRootIndex m NestedStartRoot RootIndex
n StartRoot x EndRoot NestedEndRoot &

\\\addlinespace
brief & NestRootIndex n NestStartRoot RootIndex m StartRoot x EndRoot
NestEndRoot equals NestRootIndex m NestStartRoot RootIndex n StartRoot x
EndRoot NestEndRoot &

\\\addlinespace
superbrief & NestIndex n NestRoot Index m Root x EndRoot NestEndRoot
equals NestIndex m NestRoot Index n Root x EndRoot NestEndRoot &

\\\addlinespace
\bottomrule
\end{longtable}


\E \[x^{e-2}=\sqrt{x\sqrt[3]{x\sqrt[4]{x\sqrt[5]{x\ldots}}}}, x\in\mathbb{R}\]
\begin{longtable}[c]{@{}lll@{}}
\toprule\addlinespace
verbose & x Superscript e minus 2 Baseline equals Nested3StartRoot x
NestedTwiceRootIndex 3 NestedTwiceStartRoot x NestedRootIndex 4
NestedStartRoot x RootIndex 5 StartRoot x ellipsis EndRoot NestedEndRoot
NestedTwiceEndRoot Nested3EndRoot comma x Element-of double-struck upper
R &

\\\addlinespace
brief & x Sup e minus 2 Base equals Nest3StartRoot x NestTwiceRootIndex
3 NestTwiceStartRoot x NestRootIndex 4 NestStartRoot x RootIndex 5
StartRoot x ellipsis EndRoot NestEndRoot NestTwiceEndRoot Nest3EndRoot
comma x Element-of double-struck upper R &

\\\addlinespace
superbrief & x Sup e minus 2 Base equals Nest3Root x NestTwiceIndex 3
NestTwiceRoot x NestIndex 4 NestRoot x Index 5 Root x ellipsis EndRoot
NestEndRoot NestTwiceEndRoot Nest3EndRoot comma x Element-of
double-struck upper R &

\\\addlinespace
\bottomrule
\end{longtable}


\E \[\frac{2}{\pi}=\frac{\sqrt{2}{2}}{2}\frac{\sqrt{2+\sqrt{2}}}{2}\frac{\sqrt{2+\sqrt{2+\sqrt{2}}}}{2}\ldots\]
\begin{longtable}[c]{@{}lll@{}}
\toprule\addlinespace
verbose & StartFraction 2 Over pi EndFraction equals StartFraction
StartRoot 2 EndRoot Over 2 EndFraction StartFraction NestedStartRoot 2
plus StartRoot 2 EndRoot NestedEndRoot Over 2 EndFraction StartFraction
NestedTwiceStartRoot 2 plus NestedStartRoot 2 plus StartRoot 2 EndRoot
NestedEndRoot NestedTwiceEndRoot Over 2 EndFraction ellipsis &

\\\addlinespace
brief & StartFrac 2 Over pi EndFrac equals StartFrac StartRoot 2 EndRoot
Over 2 EndFrac StartFrac NestStartRoot 2 plus StartRoot 2 EndRoot
NestEndRoot Over 2 EndFrac StartFrac NestTwiceStartRoot 2 plus
NestStartRoot 2 plus StartRoot 2 EndRoot NestEndRoot NestTwiceEndRoot
Over 2 EndFrac ellipsis &

\\\addlinespace
superbrief & Frac 2 Over pi EndFrac equals Frac Root 2 EndRoot Over 2
EndFrac Frac NestRoot 2 plus Root 2 EndRoot NestEndRoot Over 2 EndFrac
Frac NestTwiceRoot 2 plus NestRoot 2 plus Root 2 EndRoot NestEndRoot
NestTwiceEndRoot Over 2 EndFrac ellipsis &

\\\addlinespace
\bottomrule
\end{longtable}



%%% Local Variables: 
%%% mode: latex
%%% TeX-master: "mathspeak"
%%% End: 
