\documentclass{article}

\newcounter{rule}
\newcounter{example}
\makeatletter
\renewcommand\section{\setcounter{rule}{0}%
  \@startsection {section}{1}{\z@}%
  {-3.5ex \@plus -1ex \@minus -.2ex}%
  {2.3ex \@plus.2ex}%
  {\normalfont\Large\bfseries}%
}
\makeatother
\setlength{\parindent}{0pt}

\newcommand\R{\addtocounter{rule}{1}
  \setcounter{example}{0}
  \vspace{1ex}
  {\large \textbf{Rule \thesection.\therule}}%

}
\newcommand\E{\addtocounter{example}{1}
  {\textbf{Example \theexample: }}%
}

\title{MathSpeak Rules}
\author{Volker Sorge}

\begin{document}
\section{Numbers}
\label{sec:numbers}

\R
\E $\pi \approx 3.14159$

\E $102 + 2214 + 15 = 2331$

\E $59 \times 0 = 0$

\R
\E $3 - -2$

\E $-y$

\E $-32$

\R

\R
\E $t2e4$

\E $\#FF0000$

\E $0x15FF + 0x2B01 = 0x4100$

\R
\E $I,II,III,IV,V.$

\R
\E $\frac{22}{7}=3.\overline{142857}$

\end{document}

%%% Local Variables: 
%%% mode: latex
%%% TeX-master: t
%%% End: 
